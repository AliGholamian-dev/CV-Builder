\documentclass{article}
\usepackage{libertinus-otf}
\setmonofont[Scale=MatchLowercase,FakeStretch=0.9]{AnonymousPro-Regular.ttf}
\newif\ifChangeBar \ChangeBarfalse
\usepackage{fancyvrb}
\usepackage{fancyvrb-ex}
\usepackage{xurl}
\usepackage[colorlinks,linktocpage]{hyperref}
\parskip=3pt plus 1pt minus 1pt
%

\newcommand\FBoxPackage{`\textsf{fancybox}'}
\newcommand\FVrbPackage{`\textsf{fancyvrb}'}
\newcommand\cs[1]{\texttt{\textbackslash#1}}
%
%  From ltugboat.cls
%
%  Typeset the name of an environment
\providecommand\env[1]{\textsf{#1}}
\providecommand\clsname[1]{\textsf{#1}}
\providecommand\pkgname[1]{\textsf{#1}}
\providecommand\optname[1]{\textsf{#1}}
\providecommand\progname[1]{\textsf{#1}}
%
%  A list of options for a package/class
\newenvironment{optlist}{\begin{description}%
  \renewcommand\makelabel[1]{%
    \descriptionlabel{\mdseries\optname{##1}}}%
  \itemsep0.25\itemsep}%
 {\end{description}}
%
%  Utility macros
%
%  Special dashes

\makeatletter
\def\thinskip{\hskip 0.16667em\relax}
\def\endash{--}
\def\emdash{\endash-}
\def\d@sh#1#2{\unskip#1\thinskip#2\thinskip\ignorespaces}
\def\dash{\d@sh\nobreak\endash}
\def\Dash{\d@sh\nobreak\emdash}
\makeatother
\newif\ifPostScriptFonts \PostScriptFontstrue
%
\setlength{\columnseprule}{0.6pt}
%
\newif\ifmulticols
%\IfFileExists{multicol.sty}{\multicolstrue}{}
\newif\ifpstricks
\pstrickstrue
%
\title{The `\textsf{fancyvrb}' package\\Fancy Verbatims in \LaTeX}
\author{Timothy Van Zandt\\Princeton University\\Princeton -- USA\\[5mm]
%        {\footnotesize \url{tvz@Princeton.EDU}}\\[5mm]
        {Packaging and documentation by}\\
        {\footnotesize Denis Girou (CNRS/IDRIS -- France),}\\[-2mm]
        {\footnotesize Sebastian Rahtz (Elsevier -- GB)}\\[-2mm]
        {\footnotesize and}\\[-2mm]
        {\footnotesize Herbert Voß\ (FU Berlin -- DE)}
        }
\date{Version 3.7\\ \today}

\fvset{frame=single,xrightmargin=0cm,numbers=left,numbersep=3pt,commentchar=Z}

\begin{document}
\maketitle
%
\begin{abstract}
    This package provides very sophisticated facilities for reading and
  writing verbatim \TeX{} code. Users can perform common tasks like changing
  font family and size, numbering lines, framing code examples, colouring
  text and conditionally processing text.
\end{abstract}

\begin{center}
Report bugs to \url{hvoss@tug.org}
\end{center}

\vfill
Thanks to 
Claudio Beccari,
Jean-François Burnol, 
Philippe Esperet, %\url{esperet@marie.polytechnique.fr}, 
Michael Friendly, %\url{friendly@hotspur.psych.yorku.ca}, 
Mario Hassler, %\url{HASSLER@ippnv2.ipp.kfa-juelich.de}, 
Mikhail Kolodin, %\url{myke@morrigan.spb.su}, 
Andreas Matthias,
Frank Mittelbach,
Rolf Niepraschk, %\url{Rolf.Niepraschk@gmx.de}, 
Will Robertson,
Ulrich M. Schwarz, 
Thomas Siegel, % \url{siegel@aix520.informatik.uni-leipzig.de}, 
Clemens Steinke,
and 
Vladimir Volovich. %\url{vvv@vvv.vsu.ru}.
\clearpage
\tableofcontents
\clearpage
%\part{\texttt{fancyvrb} package by Timothy Van Zandt}


\part{Package \texttt{fancyvrb}}


\section{Introduction}
%
  \FVrbPackage{} is the development of the \emph{verbatim} macros of the
\FBoxPackage{} package,  Section 11 of \cite{FancyBox}. It offers six kinds
of extended functionality, compared to the standard \LaTeX{}
\textsf{verbatim} environment:

\begin{enumerate}
  \item verbatim commands can be used in footnotes,
  \item several verbatim commands are enhanced,
  \item a variety of verbatim environments are provided, with many
  parameters to change the way the contents are printed; it is also possible
  to define new customized verbatim environments,
  \item a way is provided to save and restore verbatim text and environments,
  \item there are macros to write and read files in verbatim mode, with the
  usual versatility,
  \item you can build \emph{example} environments (showing both result and
  verbatim text), with the same versatility as normal verbatim.
\end{enumerate}

  The package works by scanning a line at a time from an environment or a
file. This allows it to pre-process each line, rejecting it, removing
spaces, numbering it, etc, before going on to execute the body of the line
with the appropriate catcodes set.

\section{Verbatim material in footnotes}

  After a \cs{VerbatimFootnotes} macro declaration (to use after the
preamble), it is possible to put verbatim commands and environments
(the \LaTeX{} or \FVrbPackage{} ones) in footnotes, unlike in standard
\LaTeX:

\begin{Example}
  \VerbatimFootnotes
  We can put verbatim\footnote{\verb+_Yes!_+} text in footnotes.
\end{Example}

\section{Improved verbatim commands}
%
  The \cs{DefineShortVerb} macro allows us to define a special character as
an abbreviation to enclose verbatim text and the \cs{UndefineShortVerb}
macro suppresses the special meaning of the specified character (the same
functionalities are provided in the \LaTeX{} `\textsf{shortvrb}' package):

\fvset{xrightmargin=4.8cm}

\begin{SideBySideExample}
  \DefineShortVerb{\|}
  We can simply write \Verb+_verbatim_+
  material using a single |_delimiter_|
  \UndefineShortVerb{\|}
  \DefineShortVerb{\+}
  And we can +_change_+ the character.
\end{SideBySideExample}

  To make matters more versatile, we can nominate \emph{escape} characters
in verbatim text (using the \cs{Verb} macro or with a `shortverb' character
defined), to perform formatting or similar tasks, using the
\texttt{commandchars} parameter as shown for environments in paragraph
\ref{sec:commandchars}.

\section{Verbatim environments}

Several verbatim environments are available, each with a lot of
parameters to customize them. In the following examples we use the
\texttt{Verbatim} environment, which is the equivalent of the standard
\texttt{verbatim}. The parameters can be set globally using the \cs{fvset}
macro or in an optional argument after the start of the
environment\footnote{For clarification in this paper, note that we
generally indent each verbatim line with two
spaces.}$^,$\footnote{This mechanism uses the
`\textbf{keyval}' package from the standard \LaTeX{} graphics
distribution, written by David \textsc{Carlisle}.}.

\begin{SideBySideExample}
  \begin{Verbatim}
    First verbatim line.
    Second verbatim line.
  \end{Verbatim}
\end{SideBySideExample}

\subsection{Customization of verbatim environments}

  The appearance of verbatim environments can be changed in many and
varied ways; here we list the keys that can be set. 

\subsubsection{Comments}

\begin{optlist}
  \item[commentchar (character)]: character to define comments in the
  verbatim code, so that lines starting with this character will not be
  printed (\emph{Default: empty}).
\end{optlist}

\begin{SideBySideExample}
  \begin{Verbatim}[commentchar=!]
    A comment
    Verbatim line.
    ! A comment that you will not see
  \end{Verbatim}
\end{SideBySideExample}

  Take care to a special effect if the comment character is not the first
non blank one: it is because this character is in fact managed as the \TeX{}
comment one, that is to say that it gobble the newline character too. So, in
this case, the current line will be joined with the next one and, more, the
last one will be lost if it contain a comment, as \FVrbPackage{} print a
line only after finding it end character, which will never occured in this
case...

\begin{SideBySideExample}
  \begin{Verbatim}[commentchar=\%]
    First line. % First line
    Second.
    Third line. % Third line lost...
  \end{Verbatim}
\end{SideBySideExample}

\subsubsection{Initial characters to suppress}

\begin{optlist}
  \item[gobble (integer)]: number of characters to suppress at the beginning
  of each line (up to a maximum of 9), mainly useful when environments are
  indented
  (\emph{Default: empty} \Dash no character suppressed).
\end{optlist}

\begin{SideBySideExample}
  \begin{Verbatim}
    Verbatim line.
  \end{Verbatim}

  \begin{Verbatim}[gobble=2]
    Verbatim line.
  \end{Verbatim}

  \begin{Verbatim}[gobble=8]
    Verbatim line.
  \end{Verbatim}
\end{SideBySideExample}

\subsubsection{Customization of formatting}

\begin{optlist}
  \item[formatcom (command)]: command to execute before printing verbatim
  text
  (\emph{Default: empty}).
\end{optlist}

\begin{SideBySideExample}
  \begin{Verbatim}[formatcom=\color{red}]
    First verbatim line.
    Second verbatim line.
  \end{Verbatim}
\end{SideBySideExample}

\subsubsection{Changing individual line formatting}

  The macro \cs{FancyVerbFormatLine} defines the way each line is formatted.
Its default value is \verb+\def\FancyVerbFormatLine#1{#1}+, but we can
redefine it at our convenience (\texttt{FancyVerbLine} is the name of the
line counter):

{\fvset{fontsize=\small}
\begin{SideBySideExample}
  Z\fvset{fontsize=\normalsize}
  \renewcommand{\FancyVerbFormatLine}[1]{%
    \makebox[0cm][l]{$\Rightarrow$}#1}
  \begin{Verbatim}
    First verbatim line.
    Second verbatim line.
    Third verbatim line.
  \end{Verbatim}
\end{SideBySideExample}

\begin{SideBySideExample}
  \renewcommand{\FancyVerbFormatLine}[1]{%
    \ifodd\value{FancyVerbLine}%
      \MakeUppercase{#1}\else#1\fi}
  \begin{Verbatim}
    First verbatim line.
    Second verbatim line.
    Third verbatim line.
  \end{Verbatim}
\end{SideBySideExample}
}

\subsubsection{Fonts}

\begin{optlist}
  \item[fontfamily (family name)]: font family to use.
  \texttt{tt}, \texttt{courier} and \texttt{helvetica} are pre-defined
  (\emph{Default:~tt}).
\end{optlist}

\ifPostScriptFonts  We can guess that PostScript fonts are available

\begin{SideBySideExample}
  \begin{Verbatim}[fontfamily=helvetica]
    Verbatim line.
  \end{Verbatim}
\end{SideBySideExample}

\else
\begin{quote}
    \textbf{\large Warning!} PostScript fonts seems not available on your
  platform (we are looking for the file \texttt{times.sty}, so we will
  not show the examplaes using such PostScript fonts.
\end{quote}
\fi

\begin{optlist}
  \item[fontsize (font size)]: size of the font to use
  (\emph{Default: auto} \Dash the same as the current font). If you use the
  `\textsf{relsize}' package too, you can require a change of the size
  proportional to the current one (for instance:
  \verb+fontsize=\relsize{-2}+).
\end{optlist}

\ifPostScriptFonts  We can guess that PostScript fonts are available

\begin{SideBySideExample}
  \begin{Verbatim}[fontsize=\small]
    Verbatim line.
  \end{Verbatim}

  \begin{Verbatim}[fontfamily=courier,
                   fontsize=\large]
    Verbatim line.
  \end{Verbatim}
\end{SideBySideExample}

\else
\begin{SideBySideExample}
  \begin{Verbatim}[fontsize=\small]
    Verbatim line.
  \end{Verbatim}
\end{SideBySideExample}
\fi

\begin{optlist}
  \item[fontshape (font shape)]: font shape to use
  (\emph{Default: auto} \Dash the same as the current font).
\end{optlist}

\ifPostScriptFonts  We can guess that PostScript fonts are available

\begin{SideBySideExample}
  \begin{Verbatim}[fontfamily=courier,
                   fontshape=it]
    Verbatim line.
  \end{Verbatim}
\end{SideBySideExample}

\else

\begin{SideBySideExample}
  \begin{Verbatim}[fontshape=it]
    Verbatim line.
  \end{Verbatim}
\end{SideBySideExample}
\fi

\begin{optlist}
  \item[fontseries (series name)]: \LaTeX{} font `series' to use
  (\emph{Default: auto} \Dash the same as the current font).
\end{optlist}

\ifPostScriptFonts  We can guess that PostScript fonts are available

\begin{SideBySideExample}
  \begin{Verbatim}[fontfamily=courier,
                   fontseries=b]
    Verbatim line.
  \end{Verbatim}
\end{SideBySideExample}

\else
\begin{SideBySideExample}
  \begin{Verbatim}[fontseries=b]
    Verbatim line.
  \end{Verbatim}
\end{SideBySideExample}
\fi

\subsubsection{Types and characteristics of frames}

\begin{optlist}
  \item[frame 
  (none$\mid$leftline$\mid$topline$\mid$bottomline$\mid$lines$\mid$single)]:
  type of frame around the verbatim environment
  (\emph{Default: none} \Dash no frame). With \textsf{leftline} and
  \textsf{single} modes, a space of a length  given by the \LaTeX{}
  \cs{fboxsep} macro is added between the left vertical line and the text.
\end{optlist}

 Problem at the top of a page...

\begin{SideBySideExample}
  \begin{Verbatim}[frame=leftline]
    Verbatim line.
  \end{Verbatim}

  \begin{Verbatim}[frame=topline]
    Verbatim line.
  \end{Verbatim}

  \begin{Verbatim}[frame=bottomline]
    Verbatim line.
  \end{Verbatim}

  \begin{Verbatim}[frame=lines]
    Verbatim line.
  \end{Verbatim}

  \begin{Verbatim}[frame=single]
    Verbatim line.
  \end{Verbatim}
\end{SideBySideExample}

\begin{optlist}
  \item[framerule (dimension)]: width of the rule of the frame
  (\emph{Default: 0.4pt if framing specified}).
\end{optlist}

\begin{SideBySideExample}
  \begin{Verbatim}[frame=single,
                   framerule=1mm]
    Verbatim line.
  \end{Verbatim}
\end{SideBySideExample}

\begin{optlist}
  \item[framesep (dimension)]: width of the gap between the frame and
  the text (\emph{Default: \cs{fboxsep}}).
\end{optlist}

\begin{SideBySideExample}
  \begin{Verbatim}[frame=single,
                   framesep=5mm]
    Verbatim line.
  \end{Verbatim}
\end{SideBySideExample}

\begin{optlist}
  \item[rulecolor (color command)]: color of the frame rule, expressed
  in the standard \LaTeX{} way
  (\emph{Default: black}).
\end{optlist}

\begin{SideBySideExample}
  \begin{Verbatim}[frame=single,
                   rulecolor=\color{red}]
    Verbatim line.
  \end{Verbatim}
\end{SideBySideExample}

\begin{optlist}
  \item[fillcolor (color command)]: color used to fill the space
  between the frame and the text (its thickness is given by
  \texttt{framesep})
  (\emph{Default: none} \Dash no color).
\end{optlist}

\begin{SideBySideExample}
  \begin{Verbatim}[frame=single,
         framerule=1mm,framesep=3mm,
         rulecolor=\color{red},
         fillcolor=\color{yellow}]
    Verbatim line.
  \end{Verbatim}
\end{SideBySideExample}

\subsubsection{Label for the environment}

\begingroup
\newcommand\Vitem{\SaveVerb[aftersave={\item[\UseVerb{Vitem}]}]{Vitem}}
\DefineShortVerb{\|}

\begin{optlist}
  \Vitem|label ({[string]string})|: label(s) to print on top, bottom or
  both frame lines of the environment to describe it content
  (\emph{Default: empty} \Dash no label). 
  If the label(s) contains special characters, as a comma or an equal sign,
  it must be put inside a group. If only one string is given, it will be
  used for both top and bottom lines (if the two are printed), but if an
  optional first label is given too, this one will be used for the top line 
  and the second  one for the bottom line. Note also that, if another value
  than \textsf{topline}, \textsf{bottomline}, \textsf{lines} or
  \textsf{single} is used for the \textsf{frame} parameter, the label(s)
  will not be printed.
\end{optlist}
\endgroup

 Problem at the top of a page...

\begin{SideBySideExample}
  \fvset{gobble=2}
  \begin{Verbatim}[frame=single,
                   label=My text]
    First verbatim line.
    Second verbatim line.
  \end{Verbatim}

  \begin{Verbatim}[frame=topline,
      framesep=4mm,
      label=\fbox{\Large\emph{The code}}]
    First verbatim line.
    Second verbatim line.
  \end{Verbatim}
\end{SideBySideExample}

\begin{optlist}
  \item[labelposition (none$\mid$topline$\mid$bottomline$\mid$all)]:
  position where to print the label if one is defined, which must be
  coherent with the kind of frame chosen
  (\emph{Default: none if the label is empty, topline if one label is
  defined and all if two are defined}).
  Of course, some incompatible options (like
  \textsf{frame=topline,labelposition=bottomline}) will not print the
  label.
\end{optlist}

 Problem at the top of a page...

\begin{SideBySideExample}
  \fvset{gobble=2}
  \begin{Verbatim}[frame=single,
         framesep=2mm,
         label=Text,labelposition=all]
    First verbatim line.
    Second verbatim line.
  \end{Verbatim}

  \begin{Verbatim}[frame=lines,
         label=Text,labelposition=topline]
    First verbatim line.
    Second verbatim line.
  \end{Verbatim}
\end{SideBySideExample}

\begin{SideBySideExample}
  \begin{Verbatim}[frame=bottomline,
         framesep=3mm,
         label=\textit{Code included},
         labelposition=bottomline]
    First verbatim line.
    Second verbatim line.
  \end{Verbatim}

  \begin{Verbatim}[frame=lines,
                   framesep=3mm,
   label={[Beginning of code]End of code}]
    First verbatim line.
    Second verbatim line.
  \end{Verbatim}
\end{SideBySideExample}

\subsubsection{Line numbering}

\begin{optlist}
  \item[numbers (none$\mid$left$\mid$right)]: numbering of the verbatim lines
  (\emph{Default: none} \Dash no numbering). If requested, this numbering is
  done \emph{outside} the verbatim environment.
\end{optlist}

\begin{SideBySideExample}
  \begin{Verbatim}[gobble=2,numbers=left]
    First verbatim line.
    Second verbatim line.
  \end{Verbatim}

  \begin{Verbatim}[gobble=2,
         numbers=right,numbersep=0pt]
    First verbatim line.
    Second verbatim line.
  \end{Verbatim}
\end{SideBySideExample}

\begin{optlist}
  \item[numbersep (dimension)]: gap between numbers and verbatim lines
  (\emph{Default: 12pt}).
\end{optlist}

\begin{SideBySideExample}
  \begin{Verbatim}[gobble=2,
         numbers=left,numbersep=2pt]
    First verbatim line.
    Second verbatim line.
  \end{Verbatim}
\end{SideBySideExample}

\begin{optlist}
  \item[firstnumber (auto$\mid$last$\mid$integer)]: number of the first line
  (\emph{Default: auto} \Dash numbering starts from 1). \textsf{last} means
  that the numbering is continued from the previous verbatim environment. If
  an integer is given, its value will be used to start the numbering.
\end{optlist}

\begin{SideBySideExample}
  \fvset{gobble=2,
         numbers=left,numbersep=3pt}
  \begin{Verbatim}
    Verbatim line.
  \end{Verbatim}

  \begin{Verbatim}[firstnumber=last]
    Verbatim line.
  \end{Verbatim}

  \begin{Verbatim}[firstnumber=100]
    Verbatim line.
  \end{Verbatim}
\end{SideBySideExample}

\begin{optlist}
  \item[stepnumber (integer)]: interval at which line numbers are printed
  (\emph{Default:~1} \Dash all lines are numbered).
\end{optlist}

\begin{SideBySideExample}
  \begin{Verbatim}[gobble=2,numbers=left,
         numbersep=3pt,stepnumber=2]
    First verbatim line.
    Second verbatim line.
    Third verbatim line.
  \end{Verbatim}
\end{SideBySideExample}

  The macro \cs{theFancyVerbLine} defines the typesetting style of the
numbering, and the counter used is \texttt{FancyVerbLine}:

\begin{SideBySideExample}
  \renewcommand{\theFancyVerbLine}{%
    \textcolor{red}{\small
      8.\alph{FancyVerbLine}}}
  \begin{Verbatim}[gobble=2,
         numbers=left,numbersep=2pt]
    First verbatim line.
    Second verbatim line.
    Third verbatim line.
  \end{Verbatim}
\end{SideBySideExample}

\ifChangeBar\begin{changebar}\fi
\begin{optlist}
  \item[numberblanklines (boolean)]: to number or not the empty lines
  (really empty or containing blank characters only)
  (\emph{Default: true} \Dash all lines are numbered).
\end{optlist}

\begin{SideBySideExample}
  \begin{Verbatim}[gobble=2,numbers=left,
         numbersep=3pt,
         numberblanklines=false]
    First verbatim line.


    Second verbatim line.
  \end{Verbatim}
\end{SideBySideExample}
\ifChangeBar\end{changebar}\fi
%
\subsubsection{Selection of lines to print}

\begin{optlist}
  \item[firstline (integer)]: first line to print
  (\emph{Default: empty} \Dash all lines from the first are printed).
\end{optlist}

\begin{SideBySideExample}
  \begin{Verbatim}[gobble=2,firstline=2,
         numbers=left,numbersep=2pt]
    First verbatim line.
    Second verbatim line.
    Third verbatim line.
  \end{Verbatim}
\end{SideBySideExample}

\begin{optlist}
  \item[lastline (integer)]: last line to print
  (\emph{Default: empty} \Dash all lines until the last one are printed).
\end{optlist}

\begin{SideBySideExample}
  \begin{Verbatim}[gobble=2,lastline=1,
         numbers=left,numbersep=2pt]
    First verbatim line.
    Second verbatim line.
    Third verbatim line.
  \end{Verbatim}
\end{SideBySideExample}

Instead of specifying a firstline at which to start printing a range
of lines, you can define a start string; the start of the range is
the first line that exactly equals the string.  (The comparison is made
before any characters are gobbled off the front of the line.) Similarly for a stop
string. You can mix line-numbers and strings, e.g.\ start at
firstline, and end at a stop string.  Specifying the strings is a
bit klunky.  Initially you must define the strings with
\cs{newcommand*} as in:

\begin{SideBySideExample}[gobble=2]
  \newcommand*\FancyVerbStartString{FROM}
  \newcommand*\FancyVerbStopString{TO}
  \begin{Verbatim}
    First verbatim line.
  FROM
    Second verbatim line.
  TO
    Third verbatim line.
  \end{Verbatim}
\end{SideBySideExample}

\noindent To redefine the strings, you must use \cs{renewcommand*}.

\subsubsection{Spaces and tab characters}

\begin{optlist}
  \item[showspaces (boolean)]: print a special character representing each
  space
  (\emph{Default: false} \Dash spaces not shown).
\end{optlist}

\begin{SideBySideExample}[gobble=0]
\begin{Verbatim}[showspaces=true]
    Verbatim line.
\end{Verbatim}
\end{SideBySideExample}

  In practice, all verbatim environments have a~\texttt{*} variant, which
sets \texttt{showspaces=true}:

\begin{SideBySideExample}[gobble=0]
  \begin{Verbatim*}
    Verbatim line.
  \end{Verbatim*}
\end{SideBySideExample}

  There are also some parameters to determine the way tab characters are
interpreted (using tabs is in fact a rather old-fashioned style of coding):

\begin{optlist}
  \item[showtabs (boolean)]: explicitly show tab characters
  (\emph{Default: false} \Dash tab characters not shown).
\end{optlist}

\begin{optlist}
  \item[obeytabs (boolean)]: position characters according to the tabs
  (\emph{Default: false} \Dash tab characters are added to the current
  position).
\end{optlist}

\begin{optlist}
  \item[tabsize (integer)]: number of spaces given by a tab character
  (\emph{Default:~8}).
\end{optlist}

\subsubsection{Space between lines}

\begin{optlist}
  \item[baselinestretch (auto$\mid$dimension)]: value to give to the usual
  `baselinestretch' \LaTeX{} parameter
  (\emph{Default: auto} \Dash its current value just before the verbatim
  command).
\end{optlist}

\begin{SideBySideExample}
  \begin{Verbatim}[baselinestretch=2]
    First verbatim line.
    Second verbatim line.
  \end{Verbatim}
\end{SideBySideExample}

\subsubsection{Escape characters for inserting commands}

\begin{optlist}
  \item[commandchars (three characters)]: characters which define the
  character which starts a macro and marks the beginning and end of a group;
  thus lets us introduce \emph{escape} sequences in verbatim code. Of
  course, it is better to choose special characters which are not used in
  the verbatim text! (\emph{Default: empty}).
  \label{sec:commandchars}
\end{optlist}

\begin{SideBySideExample}
  \begin{Verbatim}[commandchars=\\\{\}]
    \textit{This is a comment}
    First verbatim line.
    \fbox{Second} verbatim line.
    \textcolor{red}{Third} verbatim line.
  \end{Verbatim}

  \begin{Verbatim}[commandchars=+\[\]]
    +textit[\textbf{Verbatim} line].
  \end{Verbatim}
\end{SideBySideExample}

\ifChangeBar\begin{changebar}\fi
  Using this way, it is also possible to put labels to be able, later, to
make reference to some lines of the verbatim environments:

\begin{SideBySideExample}
  \begin{Verbatim}[commandchars=\\\{\},
         numbers=left,numbersep=2pt]
    First verbatim line.
    Second line.\label{vrb:Important}
    Third verbatim line.
  \end{Verbatim}

    As I previously shown
  line~\ref{vrb:Important}, it is...
\end{SideBySideExample}
\ifChangeBar\end{changebar}\fi

\subsubsection{Margins}

\begin{optlist}
  \item[xleftmargin (dimension)]: indentation to add at the start of each
  line
  (\emph{Default:~0pt} \Dash no left margin).
\end{optlist}

\begin{SideBySideExample}
  \begin{Verbatim}[frame=single,
                   xleftmargin=5mm]
    Verbatim line.
  \end{Verbatim}
\end{SideBySideExample}

\begin{optlist}
  \item[xrightmargin (dimension)]: right margin to add after each line
  (\emph{Default:~0pt} \Dash no right margin).
\end{optlist}

\begin{SideBySideExample}
  \begin{Verbatim}[frame=single,
                   xrightmargin=1cm]
    Verbatim line.
  \end{Verbatim}
\end{SideBySideExample}

\begin{optlist}
  \item[resetmargins (boolean)]: reset the left margin, which is useful if
  we are inside other indented environments (\emph{Default: false} \Dash no
  reset of the margin).
\end{optlist}

{%
\DefineVerbatimEnvironment{Verbatim}{Verbatim}{gobble=0,numbers=none}
\begin{SideBySideExample}
  \begin{itemize}
    \item First item
    \begin{Verbatim}[frame=single]
  Verbatim line.
    \end{Verbatim}
    \item Second item
    \begin{Verbatim}[frame=single,
                     resetmargins=true]
  Verbatim line.
    \end{Verbatim}
  \end{itemize}
\end{SideBySideExample}
}

\subsubsection{Overfull box messages}

\begin{optlist}
  \item[hfuzz (dimension)]: value to give to the \TeX{} \cs{hfuzz} dimension
  for text to format. This can be used to avoid seeing some unimportant
  \emph{Overfull box} messages (\emph{Default:~2pt}).
\end{optlist}

\subsubsection{Page breaks}

\begin{optlist}
  \item[samepage (boolean)]: in very special circumstances, we may want to
  make sure that a verbatim environment is not broken, even if it does not
  fit on the current page. To avoid a page break, we can set the
  \texttt{samepage} parameter to \emph{true} (\emph{Default: false}).
\end{optlist}

\subsubsection{Catcode characters}

\begin{optlist}
  \item[codes (macro)]: to specify \emph{catcode} changes (\emph{Default:
  empty}).
\end{optlist}

  For instance, this allows us to include formatted mathematics in verbatim
text:

{\fvset{fontsize=\small}
\begin{SideBySideExample}
  Z\fvset{fontsize=\normalsize}
  \begin{Verbatim}[commandchars=\\\{\},
         codes={\catcode`$=3\catcode`^=7}]
    x=1/sqrt(z**2) ! $\frac{1}{\sqrt{z^2}}$
  \end{Verbatim}
\end{SideBySideExample}
}  %$


\subsubsection{Active characters}

\begin{optlist}
  \item[defineactive (macro)]: to define the effect of \emph{active}
  characters
  (\emph{Default: empty}).
\end{optlist}

  This allows us to do some devious tricks: see the example in Section
\ref{sec:VerbatimInclude} on page~\pageref{sec:VerbatimInclude}.


\subsubsection{Reference label}

\begin{optlist}
  \item[reflabel (<label>)]: A  label for use of \verb|\pageref|.
\end{optlist}

\begin{SideBySideExample}
  \begin{Verbatim}[reflabel=verb0]
    First verbatim line.
    Second verbatim line.
  \end{Verbatim}
  See the verbatim on 
  page~\pageref{verb0}.
\end{SideBySideExample}




\subsection{Different kinds of verbatim environments}

\subsubsection{Verbatim environment}

  This is the `normal' verbatim environment which we have been using up to
now.

\subsubsection{BVerbatim environment}

  This environment puts the verbatim material in a \TeX{} box. Some
parameters do not work inside this environment (notably the framing ones),
but two new ones are available:

\begin{optlist}
  \item[boxwidth (auto$\mid$dimension)]: size of the box used
  (\emph{Default: auto} \Dash the width of the longest line is used).
\end{optlist}

\begin{optlist}
  \item[baseline (b$\mid$c$\mid$t)]: position of the baseline (on the
  \texttt{baseline}, the \texttt{center} or the \texttt{top} of the box)
  (\emph{Default: b}).
\end{optlist}

\begin{SideBySideExample}
  \fvset{gobble=2}
  \begin{BVerbatim}
    First
    Second
  \end{BVerbatim}
  \begin{BVerbatim}[baseline=c]
    First
    Second
  \end{BVerbatim}
\end{SideBySideExample}

\begin{SideBySideExample}
  \begin{BVerbatim}[boxwidth=2cm]
    First
    Second
  \end{BVerbatim}
  \begin{BVerbatim}[boxwidth=2cm,
                    baseline=t]
    First
    Second
  \end{BVerbatim}
\end{SideBySideExample}

\subsubsection{LVerbatim environment}

  This environment puts verbatim material into \LaTeX{} `LR' mode (the
so-called \emph{left-to-right} mode, which in fact is the same thing that
\TeX{} itself calls \emph{restricted horizontal mode}).

\subsubsection{Personalized environments}

\ifChangeBar\begin{changebar}\fi
\VerbatimFootnotes
  It is easy to define personal customized environments. You can redefine
the existing ones using the \cs{RecustomVerbatimEnvironment} macro or create
your own ones, using the \cs{DefineVerbatimEnvironment} macro\footnote{%
\ifChangeBar\begin{changebar}\fi
For verbatim commands, the \cs{CustomVerbatimCommand} and
\cs{RecustomVerbatimCommand} macros also exist; for instance:

\noindent
\verb+\RecustomVerbatimCommand{\VerbatimInput}{VerbatimInput}{frame=lines}+
\ifChangeBar\end{changebar}\fi}.
In each case, you specify the name of the new environment, the type of
environment on which it is based, and a set of initial option values. The
options can be overridden with an optional argument in the normal way:

\begin{SideBySideExample}
  \RecustomVerbatimEnvironment
    {Verbatim}{Verbatim}
    {gobble=2,frame=single}
  \begin{Verbatim}
    First verbatim line.
    Second verbatim line.
  \end{Verbatim}
\end{SideBySideExample}
\ifChangeBar\end{changebar}\fi

\begin{SideBySideExample}
  \DefineVerbatimEnvironment%
    {MyVerbatim}{Verbatim}
    {gobble=2,numbers=left,numbersep=2mm,
     frame=lines,framerule=0.8mm}
  \begin{MyVerbatim}
    First verbatim line.
    Second verbatim line.
  \end{MyVerbatim}

  \begin{MyVerbatim}[numbers=none,
                     framerule=1pt]
    First verbatim line.
    Second verbatim line.
  \end{MyVerbatim}
\end{SideBySideExample}

\section{Saving and restoring verbatim text and environments}

  The \cs{SaveVerb} and \cs{UseVerb} macros allow us to save and restore
verbatim material. \cs{UseVerb} itself is robust:

%  \DefineShortVerb{\|}
%  \SaveVerb{Verb}|_verbatim_|
\begin{SideBySideExample}
  \DefineShortVerb{\|}
  \SaveVerb{Verb}|_verbatim_|
  I have saved \UseVerb{Verb} and reuse 
  it later as many times as I want
  \subsection*{Using \UseVerb{Verb}}
  \UseVerb{Verb}.
\end{SideBySideExample}

  This also provides a solution to putting verbatim text inside \LaTeX{}
commands which do not normally permit it: % (used package \texttt{marginnote}):

{\fvset{frame=single,xrightmargin=0cm}
\begin{Example}
  \DefineShortVerb{\|}\SaveVerb{Verb}|_OK^| \marginpar{\UseVerb{Verb}}
\end{Example}
}

  There is a useful ability to use verbatim text as the item text in a
description list (something not normally permitted in \LaTeX), using the
\texttt{aftersave} parameter:

\begin{optlist}
  \item[aftersave (macro)]: macro to dynamically save some verbatim material
  (\emph{Default: empty}).
\end{optlist}

\begin{SideBySideExample}
  \newcommand{\Vitem}{%
    \SaveVerb[aftersave={%
      \item[\UseVerb{Vitem}]}]{Vitem}}
  \DefineShortVerb{\|}
  \begin{description}
    \Vitem|\MyCommand|: my command
  \end{description}
\end{SideBySideExample}

  In the same way, we can use and restore (in normal, boxed and LR mode,
using \cs{UseVerbatim}, \cs{BUseVerbatim} and \cs{LUseVerbatim} respectively)
entire verbatim environments:

\begin{SideBySideExample}
  Z\fvset{gobble=0,numbers=none}
  \begin{SaveVerbatim}{VerbEnv}
    Verbatim line.
  \end{SaveVerbatim}
  \UseVerbatim{VerbEnv}
  and \UseVerbatim{VerbEnv}
\end{SideBySideExample}

\begin{SideBySideExample}
  Z\fvset{gobble=0,numbers=none}
  \begin{SaveVerbatim}[gobble=5]{VerbEnv}
    First
    Second
  \end{SaveVerbatim}

  \fbox{\BUseVerbatim{VerbEnv}}
  and \BUseVerbatim{VerbEnv}.

  \LUseVerbatim{VerbEnv} and
  \LUseVerbatim{VerbEnv}
\end{SideBySideExample}

\section{Writing and reading verbatim files}
\label{sec:VerbatimInclude}

  The command \cs{VerbatimInput} (the variants \cs{BVerbatimInput} and
\cs{LVerbatimInput} also exist) allows inclusion of the contents of a file
with verbatim formatting. Of course, the various parameters which we have
described for customizing can still be used:

%The file we will use for \VerbatimInput

\typeout{*************************************}
\typeout{* Created files: hello.f90, file.txt}
\typeout{* See fancyhdr.dvi for an explanation}
\typeout{*************************************}

\begin{VerbatimOut}{hello.f90}
  ! A "hello" program

  program hello
    print *,"Hello world"
  end program hello
\end{VerbatimOut}

\begin{SideBySideExample}
  Z\RecustomVerbatimCommand{\VerbatimInput}{VerbatimInput}{gobble=4}
  \fvset{fontsize=\small}
  \VerbatimInput{hello.f90}

  \fvset{frame=single,numbers=left,
         numbersep=3pt}
  \VerbatimInput{hello.f90}

  \VerbatimInput[firstline=3,
      rulecolor=\color{green}]
    {hello.f90}

  \VerbatimInput[frame=lines,
      fontshape=sl,fontsize=\footnotesize]
    {hello.f90}
\end{SideBySideExample}

  We can make use of the `defineactive' parameter to set the comment lines
  in the program text in a different style:

\begin{SideBySideExample}
  Z\RecustomVerbatimCommand{\VerbatimInput}{VerbatimInput}{gobble=4}
  \def\ExclamationPoint{\char33}
  \catcode`!=\active
  \VerbatimInput%
    [defineactive=%
      \def!{\color{cyan}\itshape
        \ExclamationPoint}]
    {hello.f90}
\end{SideBySideExample}

  It is important to note that if the contents of the file does not fit on
the page, it will be automatically broken across pages as needed (unless the
\texttt{samepage} parameter has been set to \texttt{true}).

  There is also a \verb+VerbatimOut+ environment to write verbatim text to an
output file, in the same way:

\begin{SideBySideExample}
  \begin{VerbatimOut}{file.txt}
    I write that.
    And that too.
  \end{VerbatimOut}

  \VerbatimInput[frame=single,
    numbers=left,numbersep=6pt]{file.txt}
\end{SideBySideExample}

\ifChangeBar\begin{changebar}\fi
\section{Automatic pretty printing}

   Obviously, automatic \emph{pretty printing} is outside the scope of this
package. Nevertheless, this is specially interesting for verbatim
inclusion of programming code files or fragments. In the \LaTeX{} world
(not speaking of the \emph{literate programming} way), there are software
for some special languages, as the `\textsf{C++2LaTeX}' package from Norbert
\textsc{Kiesel}, but mainly two generic ones, which use completely different 
modes (an external preprocessor written in C and a \TeX{} based solution):
the `\textsf{LGrind}'~\cite{LGrind} system, currently maintened by Michael
\textsc{Piefel}, and the `\textsf{listings}'~\cite{Listings} package from
Carsten \textsc{Heinz}.

   Future versions of \FVrbPackage{} and `\textsf{listings}' packages are
planned to cooperate, which will offer great advantages to both users of the
two actual packages, and will allow \FVrbPackage{} users to have automatic
pretty printing of programming codes.

\section{Known problems}
  
\begin{itemize}
  \item Vladimir \textsc{Volovich} \verb+<vvv@vvv.vsu.ru>+ reported that the
  special character \verb+\th+, available with T1 encoding, can't be
  included as verbatim with \FVrbPackage. It can be true for other special
  characters too.
\end{itemize}

\iffalse

\section{Thanks}

  For interesting comments and suggestions, we would like to thank specially
(alphabetic order): Philippe \textsc{Esperet}
\url{esperet@marie.polytechnique.fr}, Michael \textsc{Friendly}
\url{friendly@hotspur.psych.yorku.ca}, Rolf \textsc{Niepraschk}
\url{niepraschk@gmx.de} and for bug reports Mario \textsc{Hassler}
\relax\unskip\break
\url{HASSLER@ippnv2.ipp.kfa-juelich.de}, Mikhail \textsc{Kolodin}
\relax\unskip\break
\url{myke@morrigan.spb.su}, Andreas Matthias, Ulrich M. Schwarz, and Vladimir \textsc{Volovich}
\verb+<vvv@vvv.vsu.ru>+.


\fi
\section{Conclusion}

  There are a few other possibilities that we have not described here.
Note specially that it is possible to define a customization file
(\texttt{fancyvrb.cfg}) loaded at the end of the package, to store
definitions of your customized commands and environments and to redefine
the attributes of existing ones.


\clearpage


\part{Package \texttt{fancyvrb-ex}}
This package defines some example environments which can write input code
and output side by side or on top of each other. They are all used for this documentation of \texttt{fancyvrb}.

\section{\texttt{Example} environment}

\begin{Verbatim}[gobble=2]
  \begin{Example}
    First verbatim line.
    Second verbatim line.
    Third verbatim line.
  \end{Example}
\end{Verbatim}

\begin{Example}
  First verbatim line.
  Second verbatim line.
  Third verbatim line.
\end{Example}

\begin{Verbatim}[gobble=2]
  \begin{Example}[frame=lines,framerule=1mm,
          numbers=left]
    First verbatim line.
    Second verbatim line.
    Third verbatim line.
  \end{Example}
\end{Verbatim}

\begin{Example}[frame=lines,framerule=1mm,numbers=left]
  First verbatim line.
  Second verbatim line.
  Third verbatim line.
\end{Example}

\newpage

\section{\texttt{CenterExample} environment}

\begin{Verbatim}[gobble=2]
  \begin{CenterExample}[frame=single,
       numbers=right]
    First verbatim line.
    Second verbatim line.
    Third verbatim line.
  \end{CenterExample}
\end{Verbatim}

\begin{CenterExample}[frame=single,numbers=right]
  First verbatim line.
  Second verbatim line.
  Third verbatim line.
\end{CenterExample}


\section{\texttt{SideBySideExample} environment}

\begin{Verbatim}[gobble=2]
  \begin{SideBySideExample}[xrightmargin=5cm,
      frame=lines, numbers=left]
    First verbatim line.
    Second verbatim line.
    Third verbatim line.
  \end{SideBySideExample}
\end{Verbatim}

\begin{SideBySideExample}[xrightmargin=5cm,frame=single,numbers=left]
  First verbatim line.
  Second verbatim line.
  Third verbatim line.
\end{SideBySideExample}



\iffalse



\section{\texttt{PCenterExample} environment}

\begin{Verbatim}
  \fvset{frame=lines,framerule=0.5mm,numbers=left}

  \begin{PCenterExample}(-0.5,-0.5)(0.5,0.5)
    \setlength{\unitlength}{1cm}
    \put(0,0){\circle{1}}
  \end{PCenterExample}

  \showgrid
  \begin{PCenterExample}(-1,-1)(1,1)
    \setlength{\unitlength}{1cm}
    \put(0,0){\circle{1}}
  \end{PCenterExample}
\end{Verbatim}

{\fvset{frame=lines,framerule=0.5mm,numbers=left}
\begin{PCenterExample}(-0.5,-0.5)(0.5,0.5)
  \setlength{\unitlength}{1cm}
  \put(0,0){\circle{1}}
\end{PCenterExample}
\showgrid
\begin{PCenterExample}(-1,-1)(1,1)
   \setlength{\unitlength}{1cm}
   \put(0,0){\circle{1}}
\end{PCenterExample}
}


\section{\texttt{PSideBySideExample} environment}

\begin{Verbatim}[gobble=2]
  \fvset{frame=single,xrightmargin=5cm}
  \begin{PSideBySideExample}(-2,-1)(2,1)
    \psellipse*[linecolor=yellow](2,1)
  \end{PSideBySideExample}
  \showgrid
  \begin{PSideBySideExample}(-2,-1)(2,1)
    \psellipse[linestyle=dashed](2,1)
  \end{PSideBySideExample}
\end{Verbatim}

{\fvset{frame=single,xrightmargin=5cm}
\begin{PSideBySideExample}(-2,-1)(2,1)
  \psellipse*[linecolor=yellow](2,1)
\end{PSideBySideExample}

\showgrid
\begin{PSideBySideExample}(-2,-1)(2,1)
  \psellipse[linestyle=dashed](2,1)
\end{PSideBySideExample}
}

\begin{Verbatim}[gobble=2]
  \fvset{frame=single,xrightmargin=5cm}
  \begin{PSideBySideExample}(-2,-1)(2,1)
    \psellipse[linestyle=dashed](2,1)
  \end{PSideBySideExample}
  \begin{PSideBySideExample}[numbers=right](-2,-1)(2,1)
    \psellipse[linestyle=dotted](2,1)
  \end{PSideBySideExample}
\end{Verbatim}

{\fvset{frame=single,xrightmargin=5cm}
\begin{PSideBySideExample}(-2,-1)(2,1)
  \psellipse[linestyle=dashed](2,1)
\end{PSideBySideExample}

\begin{PSideBySideExample}[numbers=right](-2,-1)(2,1)
  \psellipse[linestyle=dotted](2,1)
\end{PSideBySideExample}
}


\fi




\begin{thebibliography}{1}
  \bibitem{FancyBox} Timothy \textsc{van Zandt},
  \textit{Documentation for `fancybox': Box tips and tricks for \LaTeX}.
  Available from \url{CTAN: macros/latex/contrib/supported/fancybox}, 1993.

  \bibitem{FancyVrb} Timothy \textsc{van Zandt},
  \textit{`fancyvrb': Fancy Verbatims in \LaTeX}.
  Available from \url{CTAN: macros/latex/contrib/supported/fancyvrb}, 1998.

  \bibitem{LGrind} Various authors (current maintainer: Michael
  \textsc{Piefel}),
  \textit{The `LGrind' package}.
  Available from \url{CTAN: support/lgrind}, 1998.

  \bibitem{Listings} Carsten \textsc{Heinz},
  \textit{The `Listings' package}.
  Available from \url{CTAN: macros/latex/contrib/supported/listings}, 1996-1997.
\end{thebibliography}








\end{document}
